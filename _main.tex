% Options for packages loaded elsewhere
\PassOptionsToPackage{unicode}{hyperref}
\PassOptionsToPackage{hyphens}{url}
%
\documentclass[
]{scrbook}
\usepackage{amsmath,amssymb}
\usepackage{lmodern}
\usepackage{iftex}
\ifPDFTeX
  \usepackage[T1]{fontenc}
  \usepackage[utf8]{inputenc}
  \usepackage{textcomp} % provide euro and other symbols
\else % if luatex or xetex
  \usepackage{unicode-math}
  \defaultfontfeatures{Scale=MatchLowercase}
  \defaultfontfeatures[\rmfamily]{Ligatures=TeX,Scale=1}
\fi
% Use upquote if available, for straight quotes in verbatim environments
\IfFileExists{upquote.sty}{\usepackage{upquote}}{}
\IfFileExists{microtype.sty}{% use microtype if available
  \usepackage[]{microtype}
  \UseMicrotypeSet[protrusion]{basicmath} % disable protrusion for tt fonts
}{}
\makeatletter
\@ifundefined{KOMAClassName}{% if non-KOMA class
  \IfFileExists{parskip.sty}{%
    \usepackage{parskip}
  }{% else
    \setlength{\parindent}{0pt}
    \setlength{\parskip}{6pt plus 2pt minus 1pt}}
}{% if KOMA class
  \KOMAoptions{parskip=half}}
\makeatother
\usepackage{xcolor}
\IfFileExists{xurl.sty}{\usepackage{xurl}}{} % add URL line breaks if available
\IfFileExists{bookmark.sty}{\usepackage{bookmark}}{\usepackage{hyperref}}
\hypersetup{
  pdftitle={Exploration of company information},
  pdfauthor={Fabio Morea},
  hidelinks,
  pdfcreator={LaTeX via pandoc}}
\urlstyle{same} % disable monospaced font for URLs
\usepackage{longtable,booktabs,array}
\usepackage{calc} % for calculating minipage widths
% Correct order of tables after \paragraph or \subparagraph
\usepackage{etoolbox}
\makeatletter
\patchcmd\longtable{\par}{\if@noskipsec\mbox{}\fi\par}{}{}
\makeatother
% Allow footnotes in longtable head/foot
\IfFileExists{footnotehyper.sty}{\usepackage{footnotehyper}}{\usepackage{footnote}}
\makesavenoteenv{longtable}
\setlength{\emergencystretch}{3em} % prevent overfull lines
\providecommand{\tightlist}{%
  \setlength{\itemsep}{0pt}\setlength{\parskip}{0pt}}
\setcounter{secnumdepth}{5}
\ifLuaTeX
  \usepackage{selnolig}  % disable illegal ligatures
\fi

\title{Exploration of company information}
\author{Fabio Morea}
\date{2022-01-29}

\begin{document}
\maketitle

{
\setcounter{tocdepth}{1}
\tableofcontents
}
documentclass: book
output:
bookdown::pdf\_document2: default
bookdown::gitbook: default

\hypertarget{scope-and-objectives}{%
\chapter{Scope and objectives}\label{scope-and-objectives}}

This notebook explores the datasets that will presumably underpin future research work for the PhD in Applied Data Science and Artificial Intelligence.

\hypertarget{background-information}{%
\section{Background information:}\label{background-information}}

Research, innovation and highly skilled people are considered to be important factors in economic and social development.
Economic support policies often include funds to support research (for example with the creation of public research infrastructures), businesses (for example with tenders to co-finance innovative projects) and the training of people with the necessary skills.

Area Science Park is a national research institution that manages a science and technology park located in Trieste (Italy). Its activities can be considered a public investment in support of research and innovation, for a value of approximately 20 million euros per year.

Currently Area is hosting 70 tenants (60 companies and 10 research centers) engaged in research activities in the fields of ICT, lifesceinces and materials. Their success (or lack of it) depends on a key - and often overlooked - asset: the community of over 1600 employees, researchers and entrepreneurs.

Area is interested in measuring the effectiveness and efficiency of its activities, focusing in particular on

\begin{itemize}
\tightlist
\item
  monitoring the economic performance of tenants,
\item
  monitoring the community of skilled workers,
\item
  comparing with similar groups, mainly at a regional or national scale, but also with respect to the science and technology parks in Austria and Slovenia.
\end{itemize}

To support research work, Area Science Park can provide some relevant datasets, cureated as a part of the project \href{https://www.innovationintelligence.it/}{\emph{innovation intelligence}}. Innovation Intelligence aims to analyze information on companies in the Friuli Venezia Giulia region, using several data sources such as the chamber of commerce, the Regional Labor Market Observatory, a rating agency, as well as surveys on samples of companies.

\hypertarget{objectives-of-future-research-work---reserch-questions}{%
\section{Objectives of future research work - reserch questions}\label{objectives-of-future-research-work---reserch-questions}}

Research questions are currently defined on a general level:

\begin{itemize}
\tightlist
\item
  are tenant companies performing better than similar companies?
\item
  how to measure similarity between two companies?
\item
  how to exthed such measure to groups of companies?
\item
  how to identify clusters or communities of companies?
\end{itemize}

\hypertarget{about-this-notebook}{%
\section{About this notebook}\label{about-this-notebook}}

The notebook is divided in 6 sections: an introduction, a section for each dataset and a final section on potential future development.

\begin{enumerate}
\def\labelenumi{\arabic{enumi}.}
\tightlist
\item
  Imprese\_FVG
\item
  Bilanci\_FVG
\item
  Rating\_FVG
\item
  CO\_FVG
\item
  Features: A basic example of sample feature selection, on a small subset, where each company is represented by 5 features
\item
  Further development: calculating the age of companies based on several dates, handling non metric features: defining a custimized similarity function to identify \emph{similar} companies and estimate distances in a multi-dimensional space.
\end{enumerate}

The notebook has been written using \textbf{R-Studio} and rendered with \href{https://bookdown.org/}{\textbf{boowdown}} package.
The main libraries used for data manipulation are are \emph{dplyr} {[}\url{https://dplyr.tidyverse.org/}{]} and \emph{ggplot2} {[}\url{https://ggplot2.tidyverse.org/}{]} from the package \emph{tidyverse} {[}\url{https://www.tidyverse.org/}{]}. A useful guide to tidverse is available online at the following address: {[}\url{https://r4ds.had.co.nz/}{]}

\begin{quote}
TODO Some parts of the notebook are higlighted as ``To Do'', to highlight potential improvements in analysis, code efficiency or need for further clarifications.
\end{quote}

\hypertarget{exploring-dataset-impresefvg}{%
\chapter*{Exploring dataset ``impreseFVG''}\label{exploring-dataset-impresefvg}}
\addcontentsline{toc}{chapter}{Exploring dataset ``impreseFVG''}

Placeholder

\hypertarget{imprese}{%
\section{imprese}\label{imprese}}

\hypertarget{metadata}{%
\section{Metadata}\label{metadata}}

\hypertarget{identifiers}{%
\section{Identifiers}\label{identifiers}}

\hypertarget{location}{%
\section{location}\label{location}}

\hypertarget{exploring-dataset-bilancifvg}{%
\chapter*{Exploring dataset ``bilanciFVG''}\label{exploring-dataset-bilancifvg}}
\addcontentsline{toc}{chapter}{Exploring dataset ``bilanciFVG''}

Placeholder

\hypertarget{exploring-dataset-ratingfvg}{%
\chapter*{Exploring dataset ``ratingFVG''}\label{exploring-dataset-ratingfvg}}
\addcontentsline{toc}{chapter}{Exploring dataset ``ratingFVG''}

Placeholder

\hypertarget{exploring-dataset-co-fvg}{%
\chapter*{Exploring dataset ``CO-FVG''}\label{exploring-dataset-co-fvg}}
\addcontentsline{toc}{chapter}{Exploring dataset ``CO-FVG''}

Placeholder

\hypertarget{introduction}{%
\chapter*{introduction}\label{introduction}}
\addcontentsline{toc}{chapter}{introduction}

Placeholder

\hypertarget{similarities-based-on-metric-features}{%
\section*{similarities based on metric features}\label{similarities-based-on-metric-features}}
\addcontentsline{toc}{section}{similarities based on metric features}

\hypertarget{custimized-distances}{%
\section*{custimized distances}\label{custimized-distances}}
\addcontentsline{toc}{section}{custimized distances}

\hypertarget{further-development}{%
\chapter*{Further development}\label{further-development}}
\addcontentsline{toc}{chapter}{Further development}

Placeholder

\hypertarget{distance}{%
\section{distance}\label{distance}}

\hypertarget{a-similarity-function-between-companies-based-on-nace-codes}{%
\section{A similarity function between companies based on NACE codes}\label{a-similarity-function-between-companies-based-on-nace-codes}}

\end{document}
